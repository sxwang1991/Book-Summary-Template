% A book summary template inspired by Jan Küster's Left Sidebar CV / https://github.com/jankapunkt/latexcv

\documentclass[11pt,a4paper]{article}	
\usepackage[utf8]{inputenc}	

% some tex-live fonts - choose your own

%\usepackage[defaultsans]{droidsans}
%\usepackage[default]{comfortaa}
%\usepackage{cmbright}
\usepackage[default]{raleway}
%\usepackage{fetamont}
%\usepackage[default]{gillius}
%\usepackage[light,math]{iwona}
%\usepackage[thin]{roboto}

% set font default
\renewcommand*\familydefault{\sfdefault} 	
\usepackage[T1]{fontenc}

\usepackage{moresize}
\usepackage{fontawesome}

\usepackage{paracol}
\usepackage[margin=1.5cm]{geometry}

\usepackage{fancyhdr}
\pagestyle{empty}
\setlength{\parindent}{0mm}
\usepackage{graphicx}
	
\usepackage{tikz}				
\usetikzlibrary{shapes, backgrounds,mindmap, trees}

\usepackage{transparent}
\usepackage{color}

\definecolor{maincol}{RGB}{ 225, 0, 0 }
\definecolor{darkcol}{RGB}{ 70, 70, 70 }
\definecolor{lightcol}{RGB}{245,245,245}

\usepackage{enumitem}
\setitemize{label={\color{maincol}\faCheck}}

\usepackage[hidelinks]{hyperref}
\include{kuestercvelements}


%============================================================================%
\begin{document}
\columnratio{0.31}
\setlength{\columnsep}{2.2em}
\setlength{\columnseprule}{4pt}
\colseprulecolor{lightcol}
\begin{paracol}{2}

\includegraphics[width=\linewidth]{bookcover.jpg}

\heading{Overview}

\small

\icontext{Male}{12}{x-chromosome}{black}\\[6pt]	
\icontext{Book}{12}{Structured Writing}{black}\\[6pt]	
\icontext{MapMarker}{12}{LPS}{black}\\[6pt]
\icontext{Calendar}{12}{2020-12-24}{black}\\[6pt]

\vspace{3em}

\begin{figure}[h]
    \centering
    \includegraphics[width=\linewidth]{figure1.jpg}
    %\caption{Graph illustrating how far you can go with the same amount of energy if invested deliberately in just one thing versus diffusion}
    \label{fig:my_label}
\end{figure}


\normalsize
\switchcolumn

\titlebox{white}{darkcol}{

\bigfont{Structured Writing}

\titletext{The problem-solution pattern}


}
\vspace{1em}

%---------------------------------------------------------------------------------------

\heading{Topic}

How to keep a good relationship with children

\heading{Problem}

Many parents do not seem to know how to keep a good relationship with their children.

\heading{Details of problem}

They do not know what activities to do or what topics to talk about with their children.

\heading{Solutions}

\begin{itemize}
    \item Have a family trip.
    \item Do fun things together.
    \item Tell stories to young children.
\end{itemize}

\heading{Evaluation}

All these will contribute to a better relationship with children.

\heading{Sample paragraph}
\begin{quote}
Many parents do not seem to know how to keep a good relationship with their children.
They do not know what activities to do or what topics to talk about with their children.
Here are a few useful tips.
Having a family trip sometimes is a wonderful way for all family members to bond together.
And you and your children can do fun things together, such as playing football or watching their favorite show.
Besides, a very good activity to bond with your young children is storytelling!
You can tell them the stories about your childhood.
All these ways will contribute to a better relationship with your children.
\end{quote}












\end{paracol}
\end{document}

