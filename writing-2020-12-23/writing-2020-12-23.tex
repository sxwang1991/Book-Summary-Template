% A book summary template inspired by Jan Küster's Left Sidebar CV / https://github.com/jankapunkt/latexcv

\documentclass[11pt,a4paper]{article}	
\usepackage[utf8]{inputenc}	

% some tex-live fonts - choose your own

%\usepackage[defaultsans]{droidsans}
%\usepackage[default]{comfortaa}
%\usepackage{cmbright}
\usepackage[default]{raleway}
%\usepackage{fetamont}
%\usepackage[default]{gillius}
%\usepackage[light,math]{iwona}
%\usepackage[thin]{roboto}

% set font default
\renewcommand*\familydefault{\sfdefault} 	
\usepackage[T1]{fontenc}

\usepackage{moresize}
\usepackage{fontawesome}

\usepackage{paracol}
\usepackage[margin=1.5cm]{geometry}

\usepackage{fancyhdr}
\pagestyle{empty}
\setlength{\parindent}{0mm}
\usepackage{graphicx}
	
\usepackage{tikz}				
\usetikzlibrary{shapes, backgrounds,mindmap, trees}

\usepackage{transparent}
\usepackage{color}

\definecolor{maincol}{RGB}{ 225, 0, 0 }
\definecolor{darkcol}{RGB}{ 70, 70, 70 }
\definecolor{lightcol}{RGB}{245,245,245}

\usepackage{enumitem}
\setitemize{label={\color{maincol}\faCheck}}

\usepackage[hidelinks]{hyperref}
\include{kuestercvelements}


%============================================================================%
\begin{document}
\columnratio{0.31}
\setlength{\columnsep}{2.2em}
\setlength{\columnseprule}{4pt}
\colseprulecolor{lightcol}
\begin{paracol}{2}

\includegraphics[width=\linewidth]{bookcover.jpg}

\heading{Overview}

\small

\icontext{Male}{12}{x-chromosome}{black}\\[6pt]	
\icontext{Book}{12}{Structured Writing}{black}\\[6pt]	
\icontext{MapMarker}{12}{LPS}{black}\\[6pt]
\icontext{Calendar}{12}{2020-12-23}{black}\\[6pt]

\vspace{3em}

\begin{figure}[h]
    \centering
    \includegraphics[width=\linewidth]{figure1.jpg}
    %\caption{Graph illustrating how far you can go with the same amount of energy if invested deliberately in just one thing versus diffusion}
    \label{fig:my_label}
\end{figure}


\normalsize
\switchcolumn

\titlebox{white}{darkcol}{

\bigfont{Structured Writing}

\titletext{x-chromosome}

%The Disciplined Pursuit of Less, NY 2011.
}
\vspace{1em}

%---------------------------------------------------------------------------------------


\heading{Topic}

How to succeed in college

% \begin{quote}
%    Essentialism is the disciplined pursuit of ‘less but better’.
% \end{quote}

\heading{Topic sentence}

There are some things you can do to succeed in college.

\heading{Three things to do}

\begin{itemize}
    \item pursuing passions
    \item seizing opportunities
    \item taking responsibilities
\end{itemize}


\heading{Concluding sentence}

You will become successful not not in college, but also in your future career.


\heading{Sample paragraph}

There some things you can do to succeed in college. 
First, pursure passion.
your passions will broaden your mind and make your life interesting.
Second, never let go of any opportunities that come your way.
College is full of unique opportunities, 
which will enable you to sample new things and meet wonderful people.
Lastly, take responsibilities.
In college you must learn to be responsible for your decisions and actions.
With the passions, the opportunities, and the ability to take responsibilities,
you will become successful not only in college, 
but also in your future career.













\end{paracol}
\end{document}

